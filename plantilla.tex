\documentclass{exam-fi}

\Assignatura{Fonaments d'Inform�tica}
\Especialitat{ET So i Imatge}
\Examen{Segon Parcial, X de Y de 200Z}
\TempsMaxim{2 hores 30 minuts}
\Revisio{La revisi� ser� el dia D.}

%%%%%%%%%%%%%%%%%%%%%%%%%%%%%%%%%%%%%%%%%%%%%%%%%%%%%%%%%%%%%%%%%%%%
\begin{document}

\capsalera

\problema{4 punts}

% Aqu� va l'enunciat del problema. 

Lorem ipsum dolor sit amet, consectetuer adipiscing elit. Cras vitae
nisl nec nunc bibendum gravida. Suspendisse lectus metus, viverra in,
eleifend sed, tincidunt vel, nisl. Proin pulvinar purus in neque.
Maecenas porta pede. Vivamus at velit ut dui sagittis convallis. Duis
risus nisl, dignissim sit amet, mollis id, ultrices sit amet, arcu.
Morbi velit lorem, dapibus ac, fermentum quis, ornare quis, lorem.
Etiam ut tortor. Nam augue. Class aptent taciti sociosqu ad lit

\begin{Verbatim}
% Codi per al problema (surt tal com a l'original)
% Comentaris amb aquest caracter
struct tCantant {
  long telefon;
  string nom;
  tTipusVeu veu;
};

...
\end{Verbatim}

\apartat{1 punt} Aquest �s un apartat amb puntuaci�...

\apartat{1.5 punts} Un altre apartat amb puntuaci�.  

\apartat{} Apartat sense puntuaci�...


\problema{1.5 punts}

Un altre problema curt.

% \apartat{asd} Fistro

% \apartat{asd} Pecador


\end{document}
